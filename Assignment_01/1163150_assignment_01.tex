\documentclass[addpoints,a4paper,ngerman,12pt,answers]{exam}
\usepackage{babel}
\usepackage[a4paper,top=2.5cm,bottom=3cm,left=2.5cm,right=2cm]{geometry}
\usepackage[utf8]{inputenc}
\usepackage[T1]{fontenc}
\usepackage{booktabs}
\usepackage{graphicx}
\usepackage{csquotes}
\usepackage{paralist}
\usepackage{amsmath}
\usepackage{pslatex}
\usepackage{pdfpages}
\usepackage{import}

%\usepackage{wasysym}
%\usepackage[math]{iwona}
%\usepackage{enumerate}
%\usepackage{array}
%\usepackage{eqnarray}
%\usepackage{longtable}

% changing keywords to german
\pointpoints{Punkt}{Punkte}
\bonuspointpoints{Bonuspunkt}{Bonuspunkte}
\renewcommand{\solutiontitle}{\noindent\textbf{Lösung:}\enspace}

\chqword{Frage}
\chpgword{Seite}
\chpword{Punkte}
\chbpword{Bonus Punkte}
\chsword{Erreicht}
\chtword{Gesamt}
% scoreboard
\hpword{Punkte:}
\hsword{Ergebnis:}
\hqword{Aufgabe:}
\htword{Summe:}


\checkboxchar{\Square}
\checkedchar{\CheckedBox}

\pagestyle{headandfoot}
\firstpageheadrule
\runningheadrule
\firstpageheader{HTW Berlin}{}{SoSe 2018}
\runningheader{HTW Berlin}{1163150 - Übung 1}{SoSe 2018}
\firstpagefooter{}{}{\thepage\,/\,\numpages}
\runningfooter{Name/MatNr.:}{}{\thepage\,/\,\numpages}

\noprintanswers

\begin{document}

\vspace*{2em}

\begin{center}
	{\Large 1163150 - Übung 1}
\end{center}

\vspace*{3em}

\makebox[\textwidth]{Name, Vorname:\enspace\hrulefill}

\vspace*{2em}

\makebox[\textwidth]{Matrikelnummer:\enspace\hrulefill}

\vspace*{2em}

\begin{itemize}
\item Abgabetermin der Übung ist der 2. Mai 2018
\item Elektronische Abgaben erfolgen grundsätzlich über die Moodle-Plattform.
\item Handgeschriebene Lösungsaufgaben können ins Fach eingeworfen werden (WH C Etage 2)
\item Für jeden Tag nach Abgabefrist werden 5 Punkte der Maximalpunktzahl abgezogen
\item Über alle Übungen hinweg besitzten Sie 3 Bonustage, die Sie für eine verspätete Abgabe ohne Punktabzug verwenden können
\end{itemize}

\vspace*{25em}

\begin{center}
	\gradetable[h][questions]
\end{center}

\clearpage

\begin{questions}
    \question Lineare Algebra

    Gegeben sind die folgenden Matrizen $A$ und $B$ sowie die Vekoten $\vec{x}$ und $\vec{y}$:

    \begin{equation*}
        A =
        \begin{pmatrix}
        4 & 4 & 5 \\
        2 & 1 & 7 \\
        4 & 8 & 3
        \end{pmatrix}
        ,
        B =
        \begin{pmatrix}
        1 & 6 \\
        3 & 1 \\
        5 & 2
        \end{pmatrix}
        ,
        C =
        \begin{pmatrix}
        1 & 4 & 4 \\
        3 & 1 & 2\\
        6 & 7 & 1
        \end{pmatrix}
        ,
        \vec{x} =
        \begin{pmatrix}
        9 \\
        5 \\
        7
        \end{pmatrix}
        ,
        \vec{y} =
        \begin{pmatrix}
        3 \\
        1 \\
        5
        \end{pmatrix}
    \end{equation*}

    \vspace{0.3cm}

    \begin{parts}
        \part[1] Berechenen Sie das Skalarprodukt $\vec{x} * \vec{y}$. Notieren Sie den vollständigen Rechenweg.

        %\begin{solutionbox}{\stretch{1}}
            % UNCOMMENT AND PUT YOUR SOLUTION HERE
        %\end{solutionbox}

        \part[0] Berechenen Sie das Produkt $\vec{x} * \vec{y}^T$. Notieren Sie den vollständigen Rechenweg.

        %\begin{solutionbox}{\stretch{1}}
            % UNCOMMENT AND PUT YOUR SOLUTION HERE
        %\end{solutionbox}

        \part[2] Berechenen Sie das Produkt $A * B$. Notieren Sie den vollständigen Rechenweg.

        %\begin{solutionbox}{\stretch{1}}
            % UNCOMMENT AND PUT YOUR SOLUTION HERE
        %\end{solutionbox}

        \part[0] Berechenen Sie das Produkt $B * A$. Notieren Sie den vollständigen Rechenweg.

        %\begin{solutionbox}{\stretch{1}}
            % UNCOMMENT AND PUT YOUR SOLUTION HERE
        %\end{solutionbox}

        \part[0] Wie können Sie die Matrizen $A, B$ modizifieren, sodass die Werte des Produkts $B * A$ den Werten des Produkts $A * B$ entsprechen. Notieren Sie den vollständigen Rechenweg.

        %\begin{solutionbox}{\stretch{1}}
            % UNCOMMENT AND PUT YOUR SOLUTION HERE
        %\end{solutionbox}

        \part[0] Berechenen Sie das Frobenius-Produkt $ \left \langle A,C \right \rangle_F $. Notieren Sie den vollständigen Rechenweg.

        %\begin{solutionbox}{\stretch{1}}
            % UNCOMMENT AND PUT YOUR SOLUTION HERE
        %\end{solutionbox}

        \part[2] Berechenen Sie das Hadamard-Produkt $A \circ C$. Notieren Sie den vollständigen Rechenweg.

        %\begin{solutionbox}{\stretch{1}}
            % UNCOMMENT AND PUT YOUR SOLUTION HERE
        %\end{solutionbox}

        \part[10] Implementieren Sie das Jupyter-Notebook \frqq linear\_algebra.ipynb \flqq.
    \end{parts}

      \vspace{0.3cm}

    \question k-Nearest-Neighbor (kNN)
    \begin{parts}
        \part[0] Implementieren Sie das Jupyter-Notebook \\ \frqq k\_nearest\_neighbor\_total.ipynb \flqq.

        \part[5] Implementieren Sie die Kreuzvalidierung im Jupyter-Notebook \\ \frqq k\_nearest\_neighbor\_cross.ipynb\flqq \hspace{0.1cm} sowie die vektorisierte Distanzberechnung im Skript \frqq k\_nearest\_neighbor\_cross.py \flqq
    \end{parts}

\end{questions}

\end{document}
